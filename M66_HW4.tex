\documentclass[11pt]{article}
\usepackage{graphicx}
\usepackage{amsthm}
\usepackage{amsmath}
\usepackage{amssymb}
\usepackage[shortlabels]{enumitem}
\usepackage[margin=1in]{geometry}
\usepackage{pgfplots}

\newcommand{\N}{\mathbb{N}}

\newenvironment{solution}
  {\renewcommand\qedsymbol{$\blacksquare$}\begin{proof}[Solution]}
  {\end{proof}}

\setlength\parindent{0pt}

\newtheorem*{observation}{Observation}
\newtheorem*{theorem}{Theorem}
\newtheorem*{claim}{Claim}
\newtheorem*{corollary}{Corollary}

\theoremstyle{definition}
\newtheorem*{definition}{Definition}

\begin{document}

	\hrule
	\begin{center}
        \textbf{MATH66: Stochastic and Numerical Methods}\hfill \textbf{Fall 2023}\newline

		{\Large Homework 4}

		David Yang
	\end{center}

\hrule

\vspace{1em}

\textit{Homework 4 Problems} \\

\begin{enumerate}
    \item \textbf{(The Lipschitz Condition): Consider the ODE \[ y' = \frac{3}{2}y^{1/3} \text{ with } y(0) = 0.\]}
    \begin{enumerate}[a)]
        \item \textbf{Apply the Forward Euler method to this problem and report what you find. Does the value of $h$ (the step size) matter? Are the values you computed a reasonable approximate solution for this IVP?}
       
        \begin{solution}
            Applying the Forward Euler method to the ODE with $y(0) = 0$ gives
            \[ y_{1} = y_0 + h \cdot f(t_0, y_0) = 0 + h \cdot 0 = 0 \]

            where we see that the value of $y$ in the next iteration is independent of the step size $h$. Consequently, since $y' = \frac{3}{2}y^{1/3}$ depends only on the $y$-variable,
            $f(t, y) = f(t, 0) = 0$ for any value of $t$. \\

            Thus, the approximate solution we get to the IVP is $\boxed{y(t) = 0}$, which is a valid solution for the given ODE. 
        \end{solution}

        \item \textbf{Find a solution to this IVP using the separation of variables technique.}
        
        \begin{solution}
        We can similarly solve the IVP using separation of variables. Dividing both sides by $y^{1/3}$, we get that
        \[ \frac{1}{y^{1/3}} y' = \frac{3}{2}. \]

        Integrating both sides, we get that
        \begin{align*}
            \int \frac{1}{y^{1/3}} \, dy &= \int \frac{3}{2} \, dt \\
            \frac{3}{2}y^{2/3} &= \frac{3}{2} t 
        \end{align*}

        Dividing both sides by $\frac{3}{2}$ and raising both sides to the power of $\frac{3}{2}$, we get the solution \[y = t^{\frac{3}{2}}.\]

        Since we do see that this solution satisfies the intial value $y(0) = 0$, we find that the solution to the IVP is $\boxed{y' = t^{\frac{3}{2}}}.$
        \end{solution}
        \item \textbf{Evaluate the $y$-derivative of $f(y) = \frac{3}{2}y^{1/3}$ and argue that this function is not Lipschitz continuous in any interval that contains $y=0$.}
        
        \begin{solution}
        The $y$-derivative of $f(y)$ is 
        \[ f'(y) = \frac{3}{2} \cdot \frac{1}{3} y^{-2/3} = \frac{1}{2y^{2/3}}.\]

        This is not Lipschitz continuous in any interval containing $y=0$ since as $y_1, y_2$ get arbitrarily close to $0$, $|f(t, y_1) - f(t, y_2)|$ grows without bound compared to $|y_1 - y_2|.$
        \end{solution}
        \item \textbf{Apply a relevant theorem discussed in class to make sense of your answers in (a) and (b). Is the Forward Euler method invalid for this problem? Why or why not?}
        \begin{solution}
        We want to make use of our local existence and uniqueness theorem for first order IVPs. However, since $f$ is not Lipschitz continuous in any interval containing $y=0$, we cannot say that there is a unique solution to some interval about $y=0$, explaining
        the two distinct solutions in (a) and (b). 

        \textcolor{red}{Euler's Method}
        \end{solution}
    \end{enumerate}

    \newpage

    \textbf{Consider the initial value ODE problem: \[y' = 2(t+1)y, \, \text{ with initial value } y(0) = 1.\] Use the separation of variables technique
    to find a solution formula for this ODE. You will use this exact solution to measure errors for the following numerical calculations. For each of the following methods listed, do the following:}

    \begin{enumerate}[a)]
        \item \textbf{Approximate the value of $y(0.1)$ by calculating one-step of each numerical method using $h = 0.1$.}
        \item \textbf{Approximate the value of $y(0.1)$ by calculating one-step of each numerical method using $h = 0.01$.}
        \item \textbf{Calculate the one-step error (local truncation error) in each approximatuion and report the order of the method.}
    \end{enumerate}

    \item \textbf{Forward Euler method}
    \item \textbf{Backward Euler method}
    \item \textbf{Trapezoid method}
    \item \textbf{Explicit trapezoid method (Heun's Method)}
    \item \textbf{Taylor-2 method}
    
    \newpage

    \item \textbf{Show the local truncation error of the Backward Euler method is $h^2$.}
    
    \begin{solution}

    We will follow the procedure in Example 6.7 in \textit{Numerical Analysis (Sauer)}.
    Consider the one-step initial value problem
    \[\begin{cases}
        y' = f(t, y) \\
        y(t_i)=w_i \\
        t \in [t_i,t_{i+1}]
    \end{cases}\]


    Assuming that $y''$ is continuous, the exact solution at $t_{i+1} = t_i + h$, by Taylor's Theorem, is
    \[ y(t_{i+1}) = y(t_i + h) = y(t_i) + hy'(t_i) + \frac{h^2}{2}y''(c) \]

    Equivalently, since $y(t_i) = w_i$ and $y'(t_i) = f(t_i, w_i)$ by definition, we can rewrite this as
    \[ y(t_{i+1}) =  w_i + hf(t_i, w_i) + \frac{h^2}{2}y''(c).\]

    On the other hand, the Backwards Euler method tells us that \[w_{i+1}= w_i + hf(t_{i+1}, w_{i+1}).\]

    The local truncation error is the difference between these two expressions:
    \begin{align*} e_{i+1} &= |w_{i+1} - y(t_{i+1})|  \\
        &= \left|\left( w_i + hf(t_{i+1}, w_{i+1}) \right) - \left( w_i + hf(t_i, w_i) + \frac{h^2}{2}y''(c)\right) \right| \\
        &= \left| h(f(t_{i+1}, w_{i+1}) - f(t_i, w_i)) + \frac{h^2}{2}y''(c)\right|
    \end{align*}
        
    Note that 
    \[ f(t_{i+1}, w_{i+1}) = f(t_i + h, w_i ) = \]
    \end{solution}
    \item \textbf{Show that the 4th order RK method $(\text{Eq } 6.50, \, \text{page } 330)$ has $h^5$ error when applied to the ODE $y'=ay.$}
\end{enumerate}

\end{document}
