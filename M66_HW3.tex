\documentclass[11pt]{article}
\usepackage{graphicx}
\usepackage{amsthm}
\usepackage{amsmath}
\usepackage{amssymb}
\usepackage[shortlabels]{enumitem}
\usepackage[margin=1in]{geometry}
\usepackage{pgfplots}

\newcommand{\N}{\mathbb{N}}

\newenvironment{solution}
  {\renewcommand\qedsymbol{$\blacksquare$}\begin{proof}[Solution]}
  {\end{proof}}

\setlength\parindent{0pt}

\newtheorem*{observation}{Observation}
\newtheorem*{theorem}{Theorem}
\newtheorem*{claim}{Claim}
\newtheorem*{corollary}{Corollary}

\theoremstyle{definition}
\newtheorem*{definition}{Definition}

\begin{document}

	\hrule
	\begin{center}
        \textbf{MATH66: Stochastic and Numerical Methods}\hfill \textbf{Fall 2023}\newline

		{\Large Homework 2}

		David Yang
	\end{center}

\hrule

\vspace{1em}

\textit{Problems from Numerical Analysis (Sauer), Chapter 3.} \\

\underline{Section 3.1 (Data and Interpolating Functions), Problem 5} \\

\underline{Section 3.1 (Data and Interpolating Functions), Problem 8} \\

\underline{Section 3.1 (Data and Interpolating Functions), Problem 12} \\

\underline{Section 3.1 (Data and Interpolating Functions), Problem 17} \\

\underline{Section 3.2 (Interpolation Error), Problem 2} \\

\underline{Section 3.2 (Interpolation Error), Problem 6} \\


\end{document}

Chapter 5.2: Newton Cotes quadrature
#4(a)
repeat #4(a) using composite trapezoid
#12*