\documentclass[11pt]{article}
\usepackage{graphicx}
\usepackage{amsthm}
\usepackage{amsmath}
\usepackage{amssymb}
\usepackage[shortlabels]{enumitem}
\usepackage[margin=1in]{geometry}
\usepackage{pgfplots}

\newcommand{\N}{\mathbb{N}}

\newenvironment{solution}
  {\renewcommand\qedsymbol{$\blacksquare$}\begin{proof}[Solution]}
  {\end{proof}}

\setlength\parindent{0pt}

\newtheorem*{observation}{Observation}
\newtheorem*{theorem}{Theorem}
\newtheorem*{claim}{Claim}
\newtheorem*{corollary}{Corollary}

\theoremstyle{definition}
\newtheorem*{definition}{Definition}

\begin{document}

	\hrule
	\begin{center}
        \textbf{MATH66: Stochastic and Numerical Methods}\hfill \textbf{Fall 2023}\newline

		{\Large Homework 2}

		David Yang
	\end{center}

\hrule

\vspace{1em}

\textit{Problems from Numerical Analysis (Sauer), Chapter 3.} \\

\underline{Section 3.1 (Data and Interpolating Functions), Problem 5} \\

\begin{enumerate}[a)]
    \item \textbf{Find a polynomial $P(x)$ of degree $3$ or less whose graph passes through the four data points $(-2, 8), (0, 4), (1, 2), (3, -2)$.}
    
    \begin{solution}
    We can construct a polynomial using Lagrange interpolation:
    \begin{align*}
         P(x) &= 8\frac{(x-0)(x-1)(x-3)}{(-2-0)(-2-1)(-2-3)} + 4\frac{(x-(-2))(x-1)(x-3)}{(0-(-2))(0-1)(0-3)} \\
         &+  2\frac{(x-(-2))(x-0)(x-3)}{(1-(-2))(1-0)(1-3)}  + (-2)\frac{(x-(-2))(x-0)(x-1)}{(3-(-2))(3-0)(3-1)}.\end{align*}
    
    This simplifies to $\boxed{P(x) = 4-2x}$.
    \end{solution}
    
    \item \textbf{Describe any other polynomials of degree 4 or less which pass through the four points in part (a).}
  
    \begin{solution}
        By the Lagrange Interpolation theorem, $P(x) = 4-2x$ is the unique polynomial of degree less than $4$, but any other polynomial that passes through the given point will be of the form
        \[ \boxed{\tilde{P}(x) = 4-2x + c(x+2)x(x-1)(x-3)} \]
    
        for any constant $c$; this polynomial is constructed from the fact that the polynomial found in (a) interpolates for the given data points.
    \end{solution}
\end{enumerate}

\newpage

\underline{Section 3.1 (Data and Interpolating Functions), Problem 8} \\

\textbf{Let $P(x)$ be the degree $9$ polynomial that takes the value $112$ at $x = 1$, takes the value $2$ at $x = 10$, and equals zero for $x = 2, \dots,9$. Calculate $P(0)$.}

\begin{solution}
    We can construct such a polynomial using Lagrange interpolation:
    \[ P(x) = 112 \frac{(x-2)\dots(x-10)}{(1-2)\dots(1-10)} + 2\frac{(x-1)\dots(x-9)}{(10-1)\dots(10-9)} + \text{[0 terms from } P(2) = \dots = P(9) = 0.\text{]} \]

    Simplifying, we find that \[P(x) = -112 \frac{(x-2)\dots(x-10)}{9!} + 2\frac{(x-1)\dots(x-9)}{9!}.\]

    Thus, plugging in $x=0$, we find that \[ P(0) = -112\frac{-(10)!}{9!} + 2\frac{-(9)!}{9!} = 1120 - 2 = \boxed{1118}.\]


\end{solution}

\vspace{1cm}

\underline{Section 3.1 (Data and Interpolating Functions), Problem 12} \\

\textbf{Can a degree $3$ polynomial intersect a degree $4$ polynomial in exactly five points? Explain.}

\begin{solution}
No. By Lagrange's Interpolation theorem, there is exactly one degree $4$ or less polynomial passing through five given points; thus, there cannot be a degree $3$ polynomial and a degree $4$ polynomial passing through the same five points.    
\end{solution}

\vspace{1cm}

\underline{Section 3.1 (Data and Interpolating Functions), Problem 17} \\

\textbf{The estimated mean atmospheric concentration of carbon dioxide in earth's atmosphere is given in the table that follows, 
in parts per million by volume. Find the degree $3$ interpolating polynomial of the data and use it to estimate the CO$_2$ concentration in
(a) $1950$ and (b) $2050$. (The actual concentration in $1950$ was $310$ ppm).} 

\begin{solution}
We find the degree $3$ interpolating polynomial using Lagrange interpolation:
\begin{align*} P(x) &= 280 \frac{(x-1850)(x-1900)(x-2000)}{(1800-1850)(1800-1900)(1800-2000)} + 283 \frac{(x-1800)(x-1900)(x-2000)}{(1850-1800)(1850-1900)(1850-2000)} \\
&+ 291 \frac{(x-1800)(x-1850)(x-2000)}{(1900-1800)(1900-1850)(1900-2000)} + 370 \frac{(x-1800)(x-1850)(x-1900)}{(2000-1800)(2000-1850)(2000-1900)}\end{align*}

Plugging in $x=1950$, we find that \begin{align*}
    P(1950) &= 280\frac{(100)(50)(-50)}{(-50) (-100) (-200)} + 283 \frac{150(50)(-50)}{(50)(-50)(-150)} \\
    &+ 291\frac{(150)(100)(-50)}{(100)(50)(-100)} + 370\frac{(150)(100)(50)}{(200)(150)(100)} \\
    &= \boxed{316 \text{ ppm}}
\end{align*}

and \begin{align*}
    P(2050) &= 280\frac{(200)(150)(50)}{(-50) (-100) (-200)} + 283 \frac{250(150)(50)}{(50)(-50)(-150)} \\
    &+ 291\frac{(250)(200)(50)}{(100)(50)(-100)} + 370\frac{(250)(200)(150)}{(200)(150)(100)} \\
    &= \boxed{465 \text{ ppm}}.
\end{align*}
\end{solution}

\newpage

\underline{Section 3.2 (Interpolation Error), Problem 2} \\

\begin{enumerate}[a)]
    \item \textbf{Given the data points $(1, 0), (2, \ln 2), (4, \ln 4)$, find the degree $2$ interpolating polynomial.}
    
    \begin{solution}
        The degree $2$ interpolating polynomial, from Lagrange's interpolation formula, is
    \begin{align*} P(x) &= 0 + \ln 2\frac{(x-1)(x-4)}{(2-1)(2-4)} + \ln 4 \frac{(x-1)(x-2)}{(4-1)(4-2)} \end{align*}
    
    Simplifying, we get \[\boxed{P(x) = -\ln 2\frac{(x-1)(x-4)}{2} + \ln 4 \frac{(x-1)(x-2)}{6}}. \]
    \end{solution}

    \item \textbf{Use the result of (a) to approximate $\ln 3$.} 
    \begin{solution}
        Plugging in $x = 3$, we get that \[P(3) =  -\ln 2\frac{(3-1)(3-4)}{2} + \ln 4 \frac{(3-1)(3-2)}{6} \approx \boxed{1.155}.\]
    \end{solution}

    \item \textbf{Use Theorem 3.3 to give an error bound for the approximation in part (b).}
    
    \begin{solution}
    By Theorem $3.3$ (the Error Bounding Theorem), we know that the interpolation error for second degree approximation $P(x)$ is 
    \[ f(x) - P(x) = \frac{(x-1)(x-2)(x-4)}{3!} f^{'''}(c)\]

    where $f(x) = \ln(x)$, and so $f'''(x) = \frac{2}{x^3}.$ Since $1 \leq c \leq 4$, we know $f'''(c) = \frac{2}{c^3} \leq 2$, and so an upper bound of our error at $x = 3$ is
    \begin{align*} |f(3) - P(3)| \leq \left|\frac{(3-1)(3-2)(3-4)}{3!} \cdot 2\right| = \boxed{\frac{1}{3}}. \end{align*}
    \end{solution}

    \item \textbf{Compare the actual error to your error bound.}
    
    \begin{solution}
    The actual error of our approximation for $\ln(3)$ is \[|\ln(3) - P(3)| \approx |\ln(3) - 1.155| \approx \boxed{0.0564},\] which is smaller than our error bound found in part (c).
    \end{solution}
\end{enumerate}

\newpage

\underline{Section 3.2 (Interpolation Error), Problem 6} \\

\textbf{Assume that the polynomial $P_5(x)$ interpolates a function $f(x)$ at the six data points $(x_i , f(x_i))$ with $x$-coordinates
$x_1 = 0, x_2 = 0.2, x_3 = 0.4, x_4 = 0.6, x_5 = 0.8, x_6 = 1.$
Assume that the interpolation error at $x = 0.3$ is $|f (0.3) - P_5(0.3)| = 0.01$. 
Estimate the new interpolation error $|f (0.3) - P_7(0.3)|$ that would result if two additional interpolation points $(x_6, y_6) = (0.1, f (0.1))$ and $(x_7, y_7) = (0.5, f (0.5))$ are added. What assumptions have you made to produce this estimate?}

\begin{solution}
We know by the Error Bounding Theorem that the interpolation error for the approximation $P_5(x)$ is approximately
\[ |f(x) - P_5(x)| = \left|\frac{x(x-0.2)(x-0.4)(x-0.6)(x-0.8)(x-1)}{6!}f^{(6)}(c_1)\right| \]

where $0 \leq c_1 \leq 1$, since we are given $6$ data points. Adding two interpolation points $(x_6, y_6) = (0.1, f(0.1))$ and $(x_7, y_7) = (0.5, f(0.5))$ gives us a slightly different error bound; we have
\[ |f(x) - P_7(x)| = \left|\frac{x(x-0.1)(x-0.2)(x-0.4)(x-0.5)(x-0.6)(x-0.8)(x-1)}{8!}f^{(8)}(c_2)\right|\]

where again $0 \leq c_2 \leq 1$. Observe that we can estimate the error of $P_7(x)$ using the relative ratios between these two error bounds:
\[ \frac{|f(x) - P_7(x)|}{|f(x) - P_5(x)|} = \left| \frac{(x-0.1)(x-0.5)}{7\cdot8}\right| \left|\frac{f^{(8)}(c_2)}{f^{(6)}(c_{1})}\right|\]

Assuming that we have no extra information about $f^{(6)}(c_1)$ and $f^{(8)}(c_2)$, we estimate the relative ratio of their errors to be
\[ \frac{|f(x) - P_7(x)|}{|f(x) - P_5(x)|} \approx \left| \frac{(x-0.1)(x-0.5)}{7\cdot8}\right|,\]

or equivalently,
\[ |f(x) - P_7(x)| \approx |f(x) - P_5(x)|\left| \frac{(x-0.1)(x-0.5)}{7\cdot8}\right|,\]

Thus, plugging in $x = 0.3$ and using the fact that $|f(0.3) - P_5(0.3)| = 0.01$, we get that
\[ |f(x) - P_7(x)| \approx 0.01 \left| \frac{(0.3-0.1)(0.3-0.5)}{56}\right| \approx \boxed{7 \times 10^{-6}}.\]

To reiterate the assumptions we used to produce this estimate, we assumed that we have no information about the bounds of the derivatives of $f$.
\end{solution}

\newpage

\textit{Problems from Numerical Analysis (Sauer), Chapter 5.} \\ 

\underline{Section 5.2 (Newton-Cotes Formulas for Numerical Integration), Problem 4(a)} \\

\textbf{Apply the composite Simpson's Rule with $m = 1, 2,$ and $4$ panels to the integral $\int_0^1 xe^x \, dx$ and report the errors. Repeat with composite trapezoid.} \\

\begin{solution}
For this problem, I adapted the DeepNote notebook code we did for N2: Quadrature. \\

For the composite Simpson's rule with $1, 2$, and $4$ panels, the reported errors were approximately
\[ \boxed{0.0026, 1.69\times 10^{-4}, 1.065\times 10^{-5}.}\]

For the composite trapezoid rule with $1$, $2$, and $4$ panels, the reported errors were approximately
\[ \boxed{0.359, 0.09175, 0.023}.\]

As expected, we see that as we add more panels, the errors get smaller, and the composite Simpson's Rule error is also smaller than the composite trapezoid rule for the same panel number; this matches our expectations since composite Simpson's is a higher order approximation.
\end{solution}

\newpage

\underline{Section 5.2 (Newton-Cotes Formulas for Numerical Integration), Problem 12} \\

\textbf{Find $c_1, c_2, c_3$ such that the rule \[ \int_0^1 f(x) \, dx \approx c_1f(0) + c_2f(0.5) + c_3f(1)\] has
degree of precision greater than one. (Hint: Substitute $f(x) = 1, x,$ and $x^2$. Do you recognize the method that results?)} \\

\begin{solution}
We will first plug in $f(x) = 1$: 
\[ \int_0^1 1 \, dx = 1 = c_1 + c_2 + c_3.\]

Next, we plug in $f(x) = x$:
\begin{align*} \int_0^1 x \, dx = \left[ \frac{x^2}{2} \right]_0^1 &= \frac{1}{2} \end{align*}

and \[ c_1f(0) + c_2f(0.5) + c_3f(1) = 0.5c_2 + c_3 = \frac{1}{2}.\]

Finally, for $f(x) = x^2$, we have
\begin{align*} \int_0^1 x^2 \, dx = \left[ \frac{x^3}{3} \right]_0^1 &= \frac{1}{3} \end{align*}

and \[ c_2f(0.5) + c_3f(1) = 0.25c_2 + c_3 = \frac{1}{3}.\]

The rule has degree of precision greater than one when these equations are satisfied; 
using the final two equations, we find that 
\[c_2 = \frac{\frac{1}{2} - \frac{1}{3}}{\frac{1}{4}} = \frac{2}{3}.\]

This tells us that $c_3 = \frac{1}{6}$, and from the first equation, we get that \[c_1 = 1-c_2 -c_3 = \frac{1}{6}.\]

Thus, we get the rule \[ \int_0^1 f(x) \, dx \approx \frac{1}{6}f(0) + \frac{2}{3}f(0.5) + \frac{1}{6}f(1)\] or equivalently,
\[ \boxed{\int_0^1 f(x) \, dx \approx \frac{f(0) + 4f(0.5) + f(1)}{6}}\]
which is simply Simpson's Rule.
\end{solution}
\end{document}
