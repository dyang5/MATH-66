\documentclass[11pt]{article}
\usepackage{graphicx}
\usepackage{amsthm}
\usepackage{amsmath}
\usepackage{amssymb}
\usepackage[shortlabels]{enumitem}
\usepackage[margin=1in]{geometry}
\usepackage{pgfplots}

\newcommand{\N}{\mathbb{N}}

\newenvironment{solution}
  {\renewcommand\qedsymbol{$\blacksquare$}\begin{proof}[Solution]}
  {\end{proof}}

\setlength\parindent{0pt}

\newtheorem*{observation}{Observation}
\newtheorem*{theorem}{Theorem}
\newtheorem*{claim}{Claim}
\newtheorem*{corollary}{Corollary}

\theoremstyle{definition}
\newtheorem*{definition}{Definition}

\begin{document}

	\hrule
	\begin{center}
        \textbf{MATH66: Stochastic and Numerical Methods}\hfill \textbf{Fall 2023}\newline

		{\Large Homework 2}

		David Yang
	\end{center}

\hrule

\vspace{1em}

\textit{Problems from Numerical Analysis (Sauer), Chapter 0.} \\

\underline{Section 0.3 (Floating Point Representation of Real Numbers), Problem 3} \\

\textbf{For which positive integers k can the number $5+2^{-k}$ be represented exactly (with no rounding error) in double precision floating point arithmetic?}
\begin{solution}First, note that the number $5$ is represented in double precision floating point as \[5 = 1.01 \times 2^2.\]

When this number $(5)$ is summed with a number of the form $2^{-k}$, rounding error will occur if and only if $1$ is added after the $52^{\text{nd}}$ bit of the mantissa, i.e. any number smaller than
\[ 0.[00000000000000000000000000000000000000000000000000001] \times 2^{2} \]

\textit{(where the bracketed mantissa contains $52$ bits)}. Thus, no rounding occur will occur as long as 
\begin{align*} 2^k &\leq 2^{-52} \times 2^{2} \\ &= 2^{-50}. \end{align*}

Thus, the number $5+2^{-k}$ will be represented exactly in double precision floating arithmetic for any positive integer $k$ from \boxed{$1$ \text{ to } $50$}. \end{solution}

\newpage

\underline{Section 0.3 (Floating Point Representation of Real Numbers), Problem 5(a)} \\

\textbf{Do the following sums by hand in IEEE double precision computer arithmetic, using the Rounding to Nearest Rule.}

\begin{enumerate}[a)] 
    \item $(1 + (2^{-51} + 2^{-53}) - 1)$

    \begin{solution}
    Note that \[2^{-51} = 0.[00000000000000000000000000000000000000000000000000010],\] \[2^{-53} = 0.[00000000000000000000000000000000000000000000000000000]1,\] 
    and so \[ 2^{-51} + 2^{-53} = 0.[00000000000000000000000000000000000000000000000000010]1.\]

    By the Rounding to the Nearest Rule, since the $53^{\text{rd}}$ bit is $1$, the $52^{\text{nd}}$ bit is $0$, so $2^{-51} + 2^{-53}$ rounds down to
    \[0.[00000000000000000000000000000000000000000000000000010] = 2^{-51}.\]

    Adding $1$ yields $1.[00000000000000000000000000000000000000000000000000010]$ and subtracting $1$ afterwards yields
    \[ \boxed{0.[00000000000000000000000000000000000000000000000000010] = 2^{-51}}.\]
    \end{solution}
\end{enumerate}



\underline{Section 0.3 (Floating Point Representation of Real Numbers), Problem 6(a)} \\

\textbf{Do the following sums by hand in IEEE double precision computer arithmetic, using the Rounding to Nearest Rule.}

\begin{enumerate}[a)] 
    \item $(1 + (2^{-51} + 2^{-52} + 2^{-54}) - 1)$
    \begin{solution}
        Note that \[2^{-51} = 0.[00000000000000000000000000000000000000000000000000010],\] \[2^{-52} = 0.[00000000000000000000000000000000000000000000000000001], \text{ and }\] 
        \[ 2^{-54} = 0.[00000000000000000000000000000000000000000000000000000]01.\]
        This means that \[ 2^{-51} + 2^{-52} + 2^{-54}= 0.[00000000000000000000000000000000000000000000000000011]01.\]
    
        By the Rounding to the Nearest Rule, since the $53^{\text{rd}}$ bit is $0$, so $2^{-51} + 2^{-52} + 2^{-54}$ rounds down to
        \[0.[00000000000000000000000000000000000000000000000000011] = 2^{-51} + 2^{-52}.\]
    
        Adding $1$ yields $1.[00000000000000000000000000000000000000000000000000011]$ and subtracting $1$ afterwards yields
        \[ \boxed{0.[00000000000000000000000000000000000000000000000000011] = 2^{-51} + 2^{-52}}.\]
        \end{solution}

\end{enumerate}

\underline{Section 0.3 (Floating Point Representation of Real Numbers), Problem 11} \\

\textbf{Does the associative law hold for IEEE computer addition?}

\begin{solution}
No. Consider $\epsilon_{\mathrm{mach}} = 2^{-52}.$ Note that
\[ \left( 1 + \frac{\epsilon_{\mathrm{mach}}}{2}\right) + \frac{\epsilon_{\mathrm{mach}}}{2} \neq 1 + \left(\frac{\epsilon_{\mathrm{mach}}}{2} + \frac{\epsilon_{\mathrm{mach}}}{2}\right).\]

This follows since $1 + \frac{\epsilon_{\mathrm{mach}}}{2}$ rounds down to $1$, so the left-hand side evaluates to $1$ whereas the right-hand side evaluates to $1 + \epsilon_{\mathrm{mach}}.$ Thus, the associative law does not hold for IEEE computer addition.
\end{solution}

\newpage

\underline{Section 0.4 (Loss of Significance), Problem 1(a)} \\

\textbf{Identify for which values of $x$ there is subtraction of nearly equal numbers, and find an alternate form that avoids the problem.} 
\begin{enumerate}[a)] 
    \item $\frac{1 - \sec{x}}{\tan^2(x)}$
\begin{solution}

Subtraction of nearly equal numbers occurs when \[1 \approx \sec(x) = \frac{1}{\cos(x)}\] 

which occurs when $\cos(x)$ is very close to $1$. Values of $x$ at which this occur include $x$ which are close to $2\pi n$ for integer $n$. \\

We want to find an alternate form of the expression \[ \frac{1 - \sec{x}}{\tan^2{x}}. \]

We can multiply both the numerator and denominator of the fraction by $1+\sec{x}$, which gives

\begin{align*}
    \frac{1 - \sec{x}}{\tan^2{x}} &= \frac{1-\sec{x}}{\tan^2{x}} \cdot \frac{1+\sec{x}}{1+\sec{x}} \\
    &= \frac{1 - \sec^2(x)}{\tan^2(x) (1 + \sec{x})}
\end{align*}

Using the identity $\tan^2(x) = \sec^2(x) - 1$, we can rewrite the numerator as $-\tan^2(x)$. Thus, we get that

\begin{align*}
    \frac{1 - \sec{x}}{\tan^2{x}} &= \frac{1 - \sec^2(x)}{\tan^2(x) (1 + \sec{x})} \\
    &= \frac{-\tan^2(x)}{\tan^2(x)(1+\sec x)} \\
    &= -\frac{1}{1+\sec{x}}.
\end{align*}

Thus, an alternate form of the expression $\frac{1 - \sec{x}}{\tan^2(x)}$ 
that avoids the potential problem of subtraction of nearly equal numbers is \[ \boxed{-\frac{1}{1+\sec{x}}} .\]
\end{solution}

\end{enumerate}

\newpage

\underline{Section 0.4 (Loss of Significance), Problem 3} \\

\textbf{Explain how to most accurately compute the two roots of the equation
$x^2+bx - 10^{-12} =0$, where $b$ is a number greater than $100$.}

\begin{solution}
The quadratic formula tells us that the roots of equation \[ x = \frac{-b \pm \sqrt{b^2 -4ac}}{2a} = \frac{-b \pm \sqrt{b^2 + 4\cdot 10^{-12}}}{2}.\]

Since $\sqrt{b^2 + 4 \cdot 10^{-12}} \approx b$ as $b \gg 10^{-12}$, the root \[ x = \frac{-b + \sqrt{b^2 + 4\cdot10^{-12}}}{2} \]

may lead to rounding error caused by the subtraction of nearly equal numbers. \\

As derived in Example 0.6, since $b$ is positive, we can use the alternate form of the quadratic formula, which gives two roots

\[ x_1 = -\frac{b + \sqrt{b^2 - 4ac}}{2a} \text { and  } x_2 = -\frac{2c}{(b + \sqrt{b^2 - 4ac})}. \]

Note that these forms avoid the rounding errors discussed above. Thus, in this instance, we can most accurately compute the two roots of the given equation by using the formulas

\[ \boxed{ x_1 = -\frac{b + \sqrt{b^2 + 4\cdot10^{-12}}}{2} \text { and  } x_2 = \frac{2\cdot 10^{-12}}{(b + \sqrt{b^2 + 4\cdot10^{-12}})} } \]
\end{solution}
\end{document}